\documentclass[11pt,a4paper,draft, fleqn, twocolumn]{article}
\usepackage[utf8]{inputenc}
\usepackage[T1]{fontenc}
\usepackage[ngerman]{babel}
\usepackage{amsmath}
\usepackage{amsfonts}
\usepackage{amssymb}
\usepackage{graphicx}
\numberwithin{equation}{section}
\setlength{\columnseprule}{0.4pt}
\setlength{\columnsep}{2em}
\begin{document}
\twocolumn[
    {\centering
        {\huge Formelsammlung Energietechnik\par}\vspace{3ex}
        {\Large Maximilian Schnadt\par}\vspace{2ex}
        \today\par}
    \tableofcontents\par
    \vspace{10ex}]
\section{Mathematik}

Eulersche Identität
\begin{equation}
    r \cdot e^{j\varphi} = r \cdot [cos(\varphi) + j \cdot sin(\varphi)]
\end{equation}

\noindent Winkel zwischen x und y Achse
\begin{equation}
\begin{split}
    tan(\varphi) & = \frac{y}{x} \Rightarrow \varphi = \arctan(\frac{y}{x})+\theta \\
    \theta & = 
    \begin{cases}
        0 & x > 0, \hspace{1ex} y > 0 \\
        \pi & x < 0, \hspace{1ex} y \neq  0 \\
        2\pi & x > 0, \hspace{1ex} y < 0
    \end{cases}
\end{split}
\end{equation}

\noindent Zeigerlänge aus Realteil und Imaginärteil
\begin{equation}
    r = \sqrt{x^2 + y^2}
\end{equation}

\noindent Realteil und Imaginärteil aus Zeigerlänge
\begin{equation}
\begin{split}
    x = r \cdot cos(\varphi) \\
    y = r \cdot sin(\varphi)
\end{split}
\end{equation}

\noindent Winkelgeschwindigkeit, Frequenz und Periodendauer
\begin{equation}
    \omega = 2\pi f = \frac{2\pi}{T}
\end{equation}

\noindent Multiplikation von Potenzen
\begin{equation}
    a^b \cdot a^c = a^{b+c}
\end{equation}

\section{Elektrotechnik}

Ohmsches Gesetz (+ im Komplexen)
\begin{equation}
    R = \frac{U}{I} \hspace{3ex}
    \underline{Z} = \frac{\underline{U}}{\underline{I}}
\end{equation}

\noindent Elektrische Leistung
\begin{equation}
    P = U \cdot I = \frac{U^2}{R} = I^2 \cdot R
\end{equation}

\noindent Elektrische Energie
\begin{equation}
    W = P \cdot t
\end{equation}

\noindent Wirkungsgrad
\begin{equation}
    \eta = \frac{P_{in}}{P_{out}}
\end{equation}

\noindent Widerstand Kondensator
\begin{equation}
    X_C = \frac{1}{j 2\pi f C}
\end{equation}

\noindent Widerstand Spule
\begin{equation}
    X_L = j 2\pi f L
\end{equation}

\pagebreak
\section{Physikalische Konstanten}

Elementarladung
\begin{equation}
    e = 1,602 \hspace{1ex} 176 \hspace{1ex} 634 \cdot 10^{-19} \text{As}
\end{equation}

\noindent Permeabilität Vakuum (magn. Feldkonstante)
\begin{equation}
    \mu_0 = 1,256 \hspace{1ex} 637 \hspace{1ex} 062 \cdot 10^{-6} \frac{\text{Vs}}{\text{Am}}
\end{equation}

\noindent Permittivität Vakuum (elektr. Feldkonstante)
\begin{equation}
    \varepsilon_0 = 8,854 \hspace{1ex} 187 \hspace{1ex} 812 \cdot 10^{-12} \frac{\text{As}}{\text{Vm}}
\end{equation}
    
\end{document}
