\documentclass[11pt, a4paper, draft, fleqn, twocolumn]{article}
\usepackage[utf8]{inputenc}
\usepackage[T1]{fontenc}
\usepackage[ngerman]{babel}
\usepackage{amsmath}
\usepackage{amsfonts}
\usepackage{amssymb}
\usepackage{graphicx}

\usepackage[linktoc=page, colorlinks, hyperfootnotes, urlcolor=black]{hyperref}
\renewcommand{\thefootnote}{\textcolor{blue}{\arabic{footnote}}}

\numberwithin{equation}{subsection}

\setlength{\columnseprule}{0.4pt}
\setlength{\columnsep}{2em}

\begin{document}
\twocolumn[
    {\centering
        {\huge Formelsammlung Energietechnik\par}\vspace{3ex}
        {\Large Alexandros Raptis, Maximilian Schnadt\par}\vspace{2ex}
        \today\par}
    \tableofcontents\par
    \vspace{10ex}]



\section{Mathematik}

\subsection{Komplexe Zahlen}

Die schönste Gleichung der Mathematik
\begin{equation}
    e^{i\pi} = -1
\end{equation}

\noindent Eulersche Identität
\begin{equation}
    r \cdot e^{j\varphi} = r \cdot [cos(\varphi) + j \cdot sin(\varphi)]
\end{equation}

\subsection{Trigonometrie}

Winkel zwischen x und y Achse
\begin{equation}
\begin{split}
    tan(\varphi) & = \frac{y}{x} \Rightarrow \varphi = \arctan(\frac{y}{x})+\theta \\
    \theta & = 
    \begin{cases}
        0 & x > 0, \hspace{1ex} y > 0 \\
        \pi & x < 0, \hspace{1ex} y \neq  0 \\
        2\pi & x > 0, \hspace{1ex} y < 0
    \end{cases}
\end{split}
\end{equation}

\noindent Zeigerlänge aus Realteil und Imaginärteil
\begin{equation}
    r = \sqrt{x^2 + y^2}
\end{equation}

\noindent Realteil und Imaginärteil aus Zeigerlänge
\begin{equation}
\begin{split}
    x = r \cdot cos(\varphi) \\
    y = r \cdot sin(\varphi)
\end{split}
\end{equation}

\noindent Winkelgeschwindigkeit, Frequenz und Periodendauer
\begin{equation}
    \omega = 2\pi f = \frac{2\pi}{T}
\end{equation}

\noindent Multiplikation von Potenzen
\begin{equation}
    a^b \cdot a^c = a^{b+c}
\end{equation}



\section{Elektrotechnik}

\subsection{Elementargesetze}

Ohmsches Gesetz (+ im Komplexen)
\begin{equation}
    R = \frac{U}{I} \hspace{3ex}
    \underline{Z} = \frac{\underline{U}}{\underline{I}}
\end{equation}

\noindent Elektrische Leistung
\begin{equation}
    P = U \cdot I = \frac{U^2}{R} = I^2 \cdot R
\end{equation}

\noindent Elektrische Energie
\begin{equation}
    W = P \cdot t
\end{equation}

\noindent Wirkungsgrad
\begin{equation}
    \eta = \frac{P_{in}}{P_{out}}
\end{equation}

\noindent Temperaturabhängigkeit des Widerstands
\begin{equation}
    R_\vartheta = R_{20}\cdot(1 + \alpha_{20} \cdot \Delta \vartheta)
\end{equation}


\subsection{Gleichstromkreise}

Reihenschaltung Widerstände
\begin{equation}
    R_{ges} = \sum_n R_n
\end{equation}

\noindent Parallelschaltung Widerstände
\begin{equation}
\begin{split}
    \frac{1}{R_{ges}} = \sum_n \frac{1}{R_n} \\
    \Rightarrow R_{ges} = \frac{1}{\sum_n \frac{1}{R_n}}
\end{split}
\end{equation}

\noindent Spannungsteiler
\begin{equation}
    \frac{U_1}{U_{ges}} = \frac{R_1}{R_{ges}}
\end{equation}

\noindent Stromteiler
\begin{equation}
    \frac{I_1}{I_2} = \frac{R_2}{R_1}
\end{equation}

\subsection{Wechselstromkreise}

Merksatz Spulen
\begin{quote}
    \textit{,,Bei Induktivitäten, die Ströme sich verspäten.''}
\end{quote}

\noindent Merksatz Kondensatoren
\begin{quote}
    \textit{,,Im Kondensator eilt der Strom vor.''}
\end{quote}

\noindent Widerstand Kondensator
\begin{equation}
    X_C = \frac{1}{j 2\pi f C}
\end{equation}

\noindent Widerstand Spule
\begin{equation}
    X_L = j 2\pi f L
\end{equation}



\section{Energietechnik}

\subsection{Leiterspannungen}

Leiterspannung
\begin{equation}
    U_L = U \cdot \sqrt{3}
\end{equation}

\subsection{Kurzschlussberechnungen}

Minimaler Kurzschlussstrom
\begin{equation}
    I_{K_{min}} = I_K \cdot c_{min}
\end{equation}

\noindent Maximaler Kurzschlussstrom
\begin{equation}
    I_{K_{max}} = I_K \cdot c_{max}
\end{equation}

\noindent Anfangskurzschlusswechselstrom
\begin{equation}
    I''_k = \dfrac{c \cdot U_n}{\sqrt{3}\cdot \lvert \underline{Z}_k \rvert}
\end{equation}

\noindent Kurzschlussimpedanz
\begin{equation}
    \lvert \underline{Z}_k \rvert = R_k + j X_k
\end{equation}

\noindent Stoßziffer
\begin{equation}
    \kappa = 1,02 + 0,98 \cdot e^{-3 \frac{R_{tot}}{X_{tot}}}
\end{equation}

\noindent Stoßkurzschlussstrom
\begin{equation}
    i_p = \sqrt{2} \kappa I''_k
\end{equation}

\noindent Verhältnis Resistanz und Reaktanz
\begin{equation}
    
\end{equation}

\noindent Netzinnenimpedanz
\begin{equation}
    Z_{Q} = c \cdot \frac{U_{nQ}^2}{S_{KQ}''}
\end{equation}



\section{Physikalische Konstanten}

Elementarladung\footnotemark[1]
\begin{equation}
    e = 1,602 \hspace{1ex} 176 \hspace{1ex} 634 \cdot 10^{-19} \text{As}
\end{equation}

\noindent Permeabilität Vakuum (magn. Feldkonstante)\footnotemark[1]
\begin{equation}
    \mu_0 = 1,256 \hspace{1ex} 637 \hspace{1ex} 062 \cdot 10^{-6} \frac{\text{Vs}}{\text{Am}}
\end{equation}

\noindent Permittivität Vakuum (elektr. Feldkonstante)\footnotemark[1]
\begin{equation}
    \varepsilon_0 = 8,854 \hspace{1ex} 187 \hspace{1ex} 812 \cdot 10^{-12} \frac{\text{As}}{\text{Vm}}
\end{equation}

\footnotetext[1]{\url{https://physics.nist.gov/cuu/pdf/wall_2018.pdf}}



\section{Symbole und Formelzeichen}
\noindent In eckigen Klammern ist, wo angemessen, das zugehörige Einheitenzeichen angegeben. \\

    \noindent
    $l$ = Länge [$m$] \\
    $A$ = Fläche [$m^2$] \\
    $t$ = Zeit [$s$] \\
    $U$ = Spannung [$V$] \\
    $I$ = Strom [$A$] \\
    $R$ = Widerstand [$\Omega$] \\
    $P$ = Leistung [$W$] \\
    $W$ = Energie [$J$] \\
    $\eta$ = Wirkungsgrad \\
    $\vartheta$ = Temperatur [$K$]\\
    $X$ = Reaktanz \\
    $Z$ = Komplexer Widerstand \\
    $S$ = Komplexe Leistung \\
    $I_k$ = Kurzschlussstrom \\
    $I''_k$ = Anfangs-Kurzschlusswechselstrom \\
    $c$ = Spannungsfaktor \\
    $U_n$ = Nennspannung \\
    $\underline{Z}_k$ = Kurzschlussimpedanz \\
    $L$ = Induktivität [$H$] \\
    $C$ = Kapazität [$F$] \\
    $Q$ = elektr. Ladung [$As$] \\
    $\kappa$ = Stoßziffer \\
    $R_{tot}$ = Totaler Wirkwiderstand \\
    $X_{tot}$ = Totaler Blindwiderstand \\
    $i_p$ = Stoßkurzschlussstrom \\
    $I_b$ = Ausschaltwechselstrom \\
    $S_k''$ = Anfangs-Kurzschlusswechselstrom- \\
    \indent \hspace{1.3em} leistung \\
    

\end{document}
