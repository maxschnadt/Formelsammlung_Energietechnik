\documentclass[11pt, a4paper, draft, fleqn, twocolumn]{article}
\usepackage[utf8]{inputenc}
\usepackage[T1]{fontenc}
\usepackage[ngerman]{babel}
\usepackage{amsmath}
\usepackage{amsfonts}
\usepackage{amssymb}
\usepackage{graphicx}

\usepackage[linktoc=page, colorlinks, hyperfootnotes, urlcolor=black]{hyperref}
\renewcommand{\thefootnote}{\textcolor{blue}{\arabic{footnote}}}

\numberwithin{equation}{subsection}

\setlength{\columnseprule}{0.4pt}
\setlength{\columnsep}{2em}

\begin{document}
\twocolumn[
    {\centering
        {\huge Formelsammlung Energietechnik\par}\vspace{3ex}
        {\Large Alexandros Raptis, Maximilian Schnadt\par}\vspace{2ex}
        \today\par}
    \tableofcontents\par
    \vspace{10ex}]



\section{Mathematik}

\subsection{Komplexe Zahlen}

Die schönste Gleichung der Mathematik
\begin{equation}
    e^{i\pi} = -1
\end{equation}

\noindent Eulersche Identität
\begin{equation}
    r \cdot e^{j\varphi} = r \cdot [\cos(\varphi) + j \cdot \sin(\varphi)]
\end{equation}

\noindent Multiplikation komplexer Zahlen
\begin{equation}
    (r_1 \cdot e^{j\varphi}) \cdot (r_2 \cdot e^{j\theta}) = (r_1 \cdot r_2) \cdot e^{j(\varphi + \theta)}
\end{equation}

\noindent Division komplexer Zahlen
\begin{equation}
    \frac{r_1 \cdot e^{j\varphi}}{r_2 \cdot e^{j\theta}} = \frac{r_1}{r_2} \cdot e^{j(\varphi - \theta)}
\end{equation}


\subsection{Trigonometrie}

Winkel zwischen x und y Achse
\begin{equation}
\begin{split}
    tan(\varphi) & = \frac{y}{x} \Rightarrow \varphi = \arctan(\frac{y}{x})+\theta \\
    \theta & = 
    \begin{cases}
        0 & x > 0, \hspace{1ex} y > 0 \\
        \pi & x < 0, \hspace{1ex} y \neq  0 \\
        2\pi & x > 0, \hspace{1ex} y < 0
    \end{cases}
\end{split}
\end{equation}

\noindent Zeigerlänge aus Realteil und Imaginärteil
\begin{equation}
    r = \sqrt{x^2 + y^2}
\end{equation}

\noindent Realteil und Imaginärteil aus Zeigerlänge
\begin{equation}
\begin{split}
    x = r \cdot \cos(\varphi) \\
    y = r \cdot \sin(\varphi)
\end{split}
\end{equation}

\noindent Winkelgeschwindigkeit, Frequenz und Periodendauer
\begin{equation}
    \omega = 2\pi f = \frac{2\pi}{T}
\end{equation}

\noindent Multiplikation von Potenzen
\begin{equation}
    a^b \cdot a^c = a^{b+c}
\end{equation}



\section{Elektrotechnik}

\subsection{Elementargesetze}

Ohmsches Gesetz (+ im Komplexen)
\begin{equation}
    R = \frac{U}{I} \hspace{3ex}
    \underline{Z} = \frac{\underline{U}}{\underline{I}}
\end{equation}

\noindent Elektrische Leistung
\begin{equation}
    P = U \cdot I = \frac{U^2}{R} = I^2 \cdot R
\end{equation}

\noindent Elektrische Energie
\begin{equation}
    W = P \cdot t
\end{equation}

\noindent Wirkungsgrad
\begin{equation}
    \eta = \frac{P_{in}}{P_{out}}
\end{equation}

\noindent Temperaturabhängigkeit des Widerstands
\begin{equation}
    R_\vartheta = R_{20}\cdot(1 + \alpha_{20} \cdot \Delta \vartheta)
\end{equation}


\subsection{Gleichstromkreise}

Reihenschaltung Widerstände
\begin{equation}
    R_{ges} = \sum_n R_n
\end{equation}

\noindent Parallelschaltung Widerstände
\begin{equation}
\begin{split}
    \frac{1}{R_{ges}} = \sum_n \frac{1}{R_n} \\
    \Rightarrow R_{ges} = \frac{1}{\sum_n \frac{1}{R_n}}
\end{split}
\end{equation}

\noindent Spannungsteiler
\begin{equation}
    \frac{U_1}{U_{ges}} = \frac{R_1}{R_{ges}}
\end{equation}

\noindent Stromteiler
\begin{equation}
    \frac{I_1}{I_2} = \frac{R_2}{R_1}
\end{equation}


\subsection{Wechselstromkreise}

Merksatz Spulen
\begin{quote}
    \textit{,,Bei Induktivitäten, die Ströme sich verspäten.''}
\end{quote}

\noindent Merksatz Kondensatoren
\begin{quote}
    \textit{,,Im Kondensator eilt der Strom vor.''}
\end{quote}

\noindent Widerstand Kondensator
\begin{equation}
    X_C = \frac{1}{j 2\pi f C}
\end{equation}

\noindent Widerstand Spule
\begin{equation}
    X_L = j 2\pi f L
\end{equation}



\section{Energietechnik}


\subsection{Grundlagen}

Minimaler Kurzschlussstrom
\begin{equation}
    I_{kmin} = \frac{c_{min} \cdot U}{\sqrt{3} \cdot R_{min}}
\end{equation}

\noindent Maximaler Kurzschlussstrom
\begin{equation}
    I_{kmax} = \frac{c_{max} \cdot U}{\sqrt{3} \cdot R_{max}}
\end{equation}

\noindent Leiterspannung
\begin{equation}
    U_L = U \cdot \sqrt{3}
\end{equation}


\subsection{Leitungserwärmung}

Stromdichte
\begin{equation}
    J = \frac{I_k}{A_L}
\end{equation}

\noindent Faktor adiabatische Erwärmung (s.u.)
\begin{equation}
    \Psi = \frac{\alpha_{20} \cdot J^2 \cdot t_k}{\gamma_{20} \cdot c}
\end{equation}

\noindent Adiabatische Erwärmung
\begin{equation}
\begin{split}
\vartheta_{Ea} = & \frac{1}{\alpha_{20}} \left[ \left( \left[ 1 + \alpha_{20}(\vartheta_{A} - \vartheta_{20}) \right] e^{\Psi} \right) -1 \right] \\
& + \vartheta_{20}
\end{split}
\end{equation}

\noindent Gesamterwärmung
\begin{equation}
    \vartheta_{E} = \vartheta_{Ea} \cdot \eta_{th}
\end{equation}


\subsection{Kurzschlussstromberechnung}

Kurzschlussimpedanz
\begin{equation}
    Z_k = X_k = X_q = U \cdot \frac{U}{S} = \frac{U^2 }{S}
\end{equation}

\noindent Maximaler Kurzschlussstrom
\begin{equation}
    I_{kmax} = \frac{c_{max} \cdot U}{\sqrt{3} \cdot Z_k}
\end{equation}

\noindent Minimaler Kurzschlussstrom
\begin{equation}
    I_{kmin} = \frac{c_{min} \cdot U}{\sqrt{3} \cdot Z_k}
\end{equation}


\subsection{Impedanzen Quelle und Trafo}

Impedanz Quelle
\begin{equation}
    Z_Q = \frac{c_{max} \cdot U_Q^2}{S_{KQ}} \cdot \frac{1}{\text{Ü}_T^2}
\end{equation}

\noindent Reaktanz Quelle
\begin{equation}
    X_Q = 0,995 \cdot Z_Q
\end{equation}

\noindent Resistanz Quelle
\begin{equation}
    R_Q = 0,1 \cdot X_Q
\end{equation}

\noindent Impedanz Quelle
\begin{equation}
    Z_Q = R_Q + jX_Q
\end{equation}

\noindent Impedanz Trafo
\begin{equation}
    Z_T = \frac{u_{KT}}{100\%} \cdot \frac{U_T^2}{S_T}
\end{equation}

\noindent Resistanz Trafo
\begin{equation}
    R_T = \frac{u_{RT}}{100\%} \cdot \frac{U_T^2}{S_T}
\end{equation}

\noindent Reaktanz Trafo
\begin{equation}
    X_T = \sqrt{Z_T^2 - R_T^2}
\end{equation}

\noindent Impedanz Trafo
\begin{equation}
    Z_T = R_T + jX_T
\end{equation}

\noindent Impedanz Quelle + Trafo
\begin{equation}
    Z_{QT} = (R_Q + R_T) + j(X_Q + X_T)
\end{equation}

\noindent Maximaler Kurzschlussstrom Quelle
\begin{equation}
    I_{KQ} = \frac{c_{max} \cdot U_Q}{\sqrt{3} \cdot Z_{QT}}
\end{equation}

\noindent Gesamter Kurzschlussstrom
\begin{equation}
    I_{K_{gesamt}} = (I_{KQ} + I_{KG} + I_{KM})
\end{equation}


\subsection{Impedanz Netz}

Impedanz Quelle Oberseite
\begin{equation}
    Z_{Q_{OS}} = \frac{c_{max} \cdot U_N}{\sqrt{3} \cdot I_{kQ}} = \frac{c_{max} \cdot U_{NQ}^2}{S_K''}
\end{equation}

\noindent Impedanz Quelle Unterseite
\begin{equation}
    Z_{Q_{US}} = Z_{Q_{OS}} \cdot \frac{1}{\text{ü}^2}
\end{equation}

\noindent Winkel Netz
\begin{equation}
    \varphi_Q = \arctan(\frac{X_Q}{R_Q})
\end{equation}

\noindent Reaktanz Quelle Unterseite
\begin{equation}
    X_{Q_{US}} = Z_{Q_{US}} \cdot \sin(\varphi_Q)
\end{equation}

\noindent Resistanz Quelle Unterseite
\begin{equation}
    R_{Q_{US}} = Z_{Q_{US}} \cdot \cos(\varphi_Q)
\end{equation}

\noindent Spannungsabfall Widerstand
\begin{equation}
    u_r = \frac{P_{cu} \cdot 100\%}{S_{rr}}
\end{equation}

\noindent Spannungsabfall
\begin{equation}
    u_x = \sqrt{u_k^2 - u_r^2}
\end{equation}

\noindent Reaktanz Trafo
\begin{equation}
    X_T = \frac{u_x}{100\%} \cdot \frac{U_{rr}^2}{S_{rr}}
\end{equation}

\noindent Resistanz Trafo
\begin{equation}
    R_T = \frac{u_r}{100\%} \cdot \frac{U_{rr}^2}{S_{rr}}
\end{equation}

\noindent 
\begin{equation}
    x_T = \frac{X_T \cdot S_{rr}}{U_{rr}^2}
\end{equation}

\noindent
\begin{equation}
    K_r = 0,95 \cdot \frac{c_{max}}{1 + 0,6 \cdot x_T}
\end{equation}

\noindent Impedanz Trafo
\begin{equation}
    Z_T = \sqrt{R_T^2 + X_T^2}
\end{equation}

\noindent Gesamtwiderstand
\begin{equation}
    R_{ges} = R_{Q_{US}} + R_{K_{2}} + R_T
\end{equation}

\noindent Gesamtreaktanz
\begin{equation}
    X_{ges} = X_{Q_{US}} + X_{K_{2}} + X_T
\end{equation}

\noindent Gesamtimpedanz
\begin{equation}
    Z_{ges} = \sqrt{R_{ges}^2 + X_{ges}^2}
\end{equation}


\subsection{Impedanz Kabel}

Widerstand Kabel
\begin{equation}
    R_{K_{2}} = R' \cdot l
\end{equation}

\noindent Reaktanz Kabel
\begin{equation}
    X_{K_{2}} = X' \cdot l
\end{equation}

\noindent Maximaler Kurzschlussstrom
\begin{equation}
    I_{kmax} = \frac{c_{max} \cdot U_{RT}}{\sqrt{3} \cdot Z_{ges}}
\end{equation}


\subsection{Staffelplan}

Vorgehen:
\begin{enumerate}
    \item Ermitteln der Leitungsabschnitte pro Zone
    \item Berechnen der Widerstände für jede Zone, also summieren über alle Produkte aus Leitungslänge und spezifischem Widerstand \\ \(X_{Da-Db} = \sum_{n=a}^{b} (l_n \cdot MK_n) \)
    \item Berechnen der Widerstände für jede Zone, also Multiplikation von Toleranz mit Widerstand (anschließenden Leitungsabschnitt beachten!) \\
    \begin{equation}
    \begin{split}
        X_{1_{Da}} & = p \cdot X_{Da} \\
        X_{2_{Da}} & = p \cdot X_{Da} + p^2 \cdot X_{Db}
    \end{split}
    \end{equation}
    \item Zeichnung erstellen
    \begin{itemize}
        \item x-Achse = Widerstand, y-Achse = Zeit
        \item Für jeden Widerstand horizontale Linie zeichnen, Höhe = Zeitraum, Länge = Widerstand
        \item Senkrechte Linie zeichnen zur Verbindung der Widerstände in einer Zone
    \end{itemize}
\end{enumerate}


\subsection{Kompensation}

Scheinleistung
\begin{equation}
    S_{ges} = \sqrt{P_{ges}^2 +Q_{ges}^2}
\end{equation}

\noindent Korrigierte Leistung
\begin{equation}
    S = \frac{P}{\cos(\varphi)}
\end{equation}

\noindent Zugeführte Leistung
\begin{equation}
    P_{zu} = \frac{P_{ab}}{\eta}
\end{equation}

\noindent Reaktanz
\begin{equation}
    Q = P \cdot \tan(\varphi)
\end{equation}

\noindent Winkel
\begin{equation}
    \tan(\varphi) = \frac{\Sigma Q}{\Sigma P}
\end{equation}

\noindent Reaktanz
\begin{equation}
    Q_1 = S_1 \cdot \sin(\varphi)
\end{equation}

\noindent Reaktanz
\begin{equation}
    Q_C = Q_1 - Q_2
\end{equation}

\noindent Reaktanz
\begin{equation}
    Q = P \cdot (\tan(\varphi_1) - \tan(\varphi_2))
\end{equation}

\noindent Strom
\begin{equation}
    I_c = \frac{Q}{U_c}
\end{equation}

\noindent Strom
\begin{equation}
    I_c = \frac{U}{X_c} = U \cdot \omega \cdot C
\end{equation}

\noindent Kapazität
\begin{equation}
    C = \frac{I_c}{U \cdot \omega}
\end{equation}

\noindent Kapazität
\begin{equation}
    C = \frac{Q_c}{2 \cdot \pi \cdot f \cdot U^2}
\end{equation}

\noindent Strom
\begin{equation}
    I = \frac{P_{ges}}{U \cdot \cos(\varphi)}
\end{equation}


\subsection{Transformator}

Übersetzungsfaktor
\begin{equation}
    \text{ü} = \frac{U_{10}}{U_{20}}
\end{equation}

\noindent Leistung
\begin{equation}
    S = U_{10} \cdot I_{10}
\end{equation}

\noindent Winkel
\begin{equation}
    \cos(\varphi) = \frac{P_{0}}{S_{0}}
\end{equation}

\noindent Gesamter relativer Spannungsabfall
\begin{equation}
    u_{k_{ges}} = \frac{\sum_n S_{T_{n}}}{\sum_n \frac{S_{T_{n}}}{u_{k_{n}}}}
\end{equation}

\noindent Teilleistung Trafo
\begin{equation}
    S_i' = S_{N_{i}} \cdot \frac{u_{k_{ges}}}{U_{k_{i}}} \cdot \frac{\sum S'}{\sum S_N'}
\end{equation}

\noindent Maximale Scheinleistung
\begin{equation}
    S_{max} = u_k \cdot \left(\sum_j \frac{S_j}{U_{k_{j}}} \right)
\end{equation}


\subsection{Verlustströme}

Eisenverluststrom
\begin{equation}
    I_{fe} = I_{10} \cdot \cos(\varphi)
\end{equation}

\noindent Magnetisierungsverluststrom
\begin{equation}
    I_{mag} = I_{10} \cdot \sin(\varphi)
\end{equation}

\noindent Relativer Leerlaufstrom
\begin{equation}
    I_0 = \frac{I_{10} (\text{bei} \quad U_{1N})}{I_{1N}}
\end{equation}

\noindent Eisenverlustwiderstand
\begin{equation}
    R_{fe} = \frac{U_{10}}{I_{fe}}
\end{equation}

\noindent Hauptinduktivität
\begin{equation}
    L_h = \frac{X_h}{2 \cdot \pi \cdot f} = \frac{U_{10}}{2 \cdot \pi \cdot I_{mag1}}
\end{equation}


\subsection{Kurzschlussleistung}

Leistung
\begin{equation}
    S = U_{1K} \cdot I_{1K}
\end{equation}

\noindent Winkel
\begin{equation}
    \cos(\varphi) = \frac{P_k}{S_k}
\end{equation}

\noindent Impedanz
\begin{equation}
    Z_k = \frac{U_{1k}}{I_{1k}}
\end{equation}


\subsection{Dauerkurzschlussstrom}

Gesamtwiderstand
\begin{equation}
    (R_1 + R_2') = Z_k \cdot \cos(\varphi_k)
\end{equation}

\noindent Näherungsweiser Widerstand
\begin{equation}
\begin{split}
    & \text{für} \quad R_1 \approx R_2' \quad \text{gilt:} \\
    & R_1 \approx \frac{1}{2}(R_1 + R_2') \quad \text{und} \\
    & R_2' \approx \frac{1}{\text{ü}^2}\cdot R_1
\end{split}
\end{equation}


\subsection{Streuinduktivitäten}

Gesamtreaktanz
\begin{equation}
    (X_{S1} + X_{S2}') = Z_k \cdot \sin(\varphi_k)
\end{equation}

\noindent Näherungsweise Reaktanz
\begin{equation}
\begin{split}
    & \text{für} \quad X_{S1} \approx X_{S2}' \quad \text{gilt:} \\
    & X_{S1} \approx \frac{1}{2}(X_{S1} + X_{S2}')
\end{split}
\end{equation}

\noindent Induktivität
\begin{equation}
    L_{S1} = \frac{X_{S1}}{2 \cdot \pi \cdot f}
\end{equation}

\noindent Induktivität
\begin{equation}
    L_{S2} \approx \frac{1}{\text{ü}^2} \cdot L_{S2}'
\end{equation}


\subsection{Wirkungsgrad}

Abgegebene Leistung
\begin{equation}
    P_{ab} = \frac{1}{2} \cdot S_N \cdot \cos(\varphi_N)
\end{equation}

\noindent Strom
\begin{equation}
    I_1 = \frac{1}{2} \cdot I_{1N}
\end{equation}

\noindent Kupferverlustleistung
\begin{equation}
    P_{cu} = \frac{1}{I_{1N}^2} \cdot P_{KN}
\end{equation}

\noindent Eisenverlustleistung
\begin{equation}
    P_{fe} = P_{ON}
\end{equation}

\noindent Wirkungsgrad
\begin{equation}
    \eta = \frac{P_{ab}}{P_{ab} + P_{cu} + P_{fe}}
\end{equation}

\noindent Relativer Spannungsabfall
\begin{equation}
    u_k = \frac{U_{1k}}{U_{1N}} \cdot 100\%
\end{equation}

\noindent Spannung
\begin{equation}
    U_{1k} = \frac{U_k \cdot U_{1N}}{100\%}
\end{equation}

\noindent Strom
\begin{equation}
    I_{1N} = \frac{S_N}{U_{1N}}
\end{equation}

\noindent Eisenverlustleistung
\begin{equation}
    P_{vfe} = \frac{U_{2N}^2}{U_{20}^2} \cdot P_{20}
\end{equation}

\noindent Kupferverlustleistung
\begin{equation}
    P_{vcu} = \frac{9}{16} \cdot U_{1k} \cdot \cos(\varphi_k)
\end{equation}

\noindent Wirkungsgrad
\begin{equation}
    \eta = \frac{\text{Nennlast} \cdot P_{vfe} \cdot \cos(\varphi)}{\text{Nennlast} \cdot P_{kfe} \cdot \cos(\varphi) + P_{vfe} + P_{vcu}}
\end{equation}

\noindent Gesamtleistung
\begin{equation}
    S_{ges} = S_1 + S_2 + S_3
\end{equation}

\noindent Relativer Gesamtspannungsabfall
\begin{equation}
    u_{k_{ges}} = \frac{S_{ges}}{\sum_{G=1}^{3} \frac{S_{N_{G}}}{u_{k_{G}}}}
\end{equation}

\noindent Leistung 1
\begin{equation}
    S_1 = \frac{U_{k_{ges}}}{U_{k_{1}}} \cdot S_{N1}
\end{equation}

\noindent Leistung 2
\begin{equation}
    S_2 = \frac{U_{k_{ges}}}{U_{k_{2}}} \cdot S_{N2}
\end{equation}

\noindent Leistung 3
\begin{equation}
    S_3 = \frac{U_{k_{ges}}}{U_{k_{3}}} \cdot S_{N3}
\end{equation}

\noindent Kontrollrechnung
\begin{equation}
    S_{ges} = S_1 + S_2 + S_3
\end{equation}


\subsection{Symmetrische Komponenten}

Phasenverschiebungsfaktor 1
\begin{equation}
    \underline{a} = e^{j120\text{°}}
\end{equation}

\noindent Phasenverschiebungsfaktor 2
\begin{equation}
    \underline{a}^2 = e^{-j120\text{°}}
\end{equation}

\noindent Gesamtphasenverschiebung
\begin{equation}
    1 + \underline{a} + \underline{a}^2 = 0
\end{equation}

\noindent Spannung Mitkomponente
\begin{equation}
    \underline{U}_M = \frac{1}{3}(\underline{U}_R + \underline{a} \cdot \underline{U}_s + \underline{a}^2 \cdot \underline{U}_T)
\end{equation}

\noindent Spannung Gegenkomponente
\begin{equation}
    \underline{U}_G = \frac{1}{3}(\underline{U}_R + \underline{a}^2 \cdot \underline{U}_s + \underline{a} \cdot \underline{U}_T)
\end{equation}

\noindent Spannung Nullkomponente
\begin{equation}
    \underline{U}_N = \frac{1}{3}(\underline{U}_R + \underline{U}_s + \underline{U}_T)
\end{equation}

\noindent Strom Mitkomponente
\begin{equation}
    \underline{I}_M = \frac{1}{3}(\underline{I}_R + \underline{a} \cdot \underline{I}_s + \underline{a}^2 \cdot \underline{I}_T)
\end{equation}

\noindent Strom Gegenkomponente
\begin{equation}
    \underline{I}_G = \frac{1}{3}(\underline{I}_R + \underline{a}^2 \cdot \underline{I}_s + \underline{a} \cdot \underline{I}_T)
\end{equation}

\noindent Strom Nullkomponente
\begin{equation}
    \underline{I}_N = \frac{1}{3}(\underline{I}_R + \underline{I}_s + \underline{I}_T)
\end{equation}



\section{Formelzeichen und Symbole}

Lateinische Buchstaben
\begin{itemize}
    \item \(A_L\) = Leitungsfläche
    \item \(c_{max}\) = Maximaler Spannungsfaktor
    \item \(c_{min}\) = Minimaler Spannungsfaktor
    \item \(c\) = Spezifische Wärmekapazität
    \item \(D_a\) = Zone \(a\)
    \item \(I_0\) = Relativer Leerlaufstrom
    \item \(I_{fe}\) = Eisenverluststrom
    \item \(I_{k_{max}}\) = Maximaler Kurzschlussstrom
    \item \(I_{k_{min}}\) = Minimaler Kurzschlussstrom
    \item \(I_{mag}\) = Magnetisierungsverluststrom
    \item \(J\) = Stromdichte
    \item \(K_r\) =
    \item \(L_h\) = Hauptinduktivität
    \item \(P\) = Leistung
    \item \(R_{K/Q/T}\) = Widerstand Kabel / Quelle / Trafo
    \item \(S\) = Scheinleistung
    \item \(u_{KT/x}\) = Relativer Spannungsabfall
    \item \(u_{r/RT}\) = Relativer ohmscher Spannungsabfall
    \item \(x_T\)
    \item \(X_{K/Q/T}\) = Reaktanz Kabel / Quelle/ Trafo
    \item \(Z_{K/Q/T}\) = Impedanz Kabel / Quelle / Trafo
\end{itemize}

\noindent Griechische Buchstaben
\begin{itemize}
    \item \(\alpha_{20}\) = Temperaturkoeffizient 20 °C
    \item \(\gamma_{20}\) = Spezifische Leitfähigkeit 20 °C
    \item \(\eta\) = Wirkungsgrad
    \item \(\vartheta\) = Temperatur
    \item \(\vartheta_{20}\) = 20 °C
    \item \(\vartheta_{E}\) = Gesamterwärmung
    \item \(\vartheta_{Ea}\) = Adiabatische Erwärmung
    \item \(\Psi\) = Faktor adiabatische Erwärmung
\end{itemize}

\end{document}
