\documentclass[11pt, a4paper, draft, fleqn, twocolumn]{article}
\usepackage[utf8]{inputenc}
\usepackage[T1]{fontenc}
\usepackage[ngerman]{babel}
\usepackage{amsmath}
\usepackage{amsfonts}
\usepackage{amssymb}
\usepackage{graphicx}

\usepackage[linktoc=page, colorlinks, hyperfootnotes, urlcolor=black]{hyperref}
\renewcommand{\thefootnote}{\textcolor{blue}{\arabic{footnote}}}

\numberwithin{equation}{subsection}

\setlength{\columnseprule}{0.4pt}
\setlength{\columnsep}{2em}

\begin{document}
\twocolumn[
    {\centering
        {\huge Formelsammlung Energietechnik\par}\vspace{3ex}
        {\Large Alexandros Raptis, Maximilian Schnadt\par}\vspace{2ex}
        \today\par}
    \tableofcontents\par
    \vspace{10ex}]



\section{Mathematik}

\subsection{Komplexe Zahlen}

Die schönste Gleichung der Mathematik
\begin{equation}
    e^{i\pi} = -1
\end{equation}

\noindent Eulersche Identität
\begin{equation}
    r \cdot e^{j\varphi} = r \cdot [cos(\varphi) + j \cdot sin(\varphi)]
\end{equation}


\subsection{Trigonometrie}

Winkel zwischen x und y Achse
\begin{equation}
\begin{split}
    tan(\varphi) & = \frac{y}{x} \Rightarrow \varphi = \arctan(\frac{y}{x})+\theta \\
    \theta & = 
    \begin{cases}
        0 & x > 0, \hspace{1ex} y > 0 \\
        \pi & x < 0, \hspace{1ex} y \neq  0 \\
        2\pi & x > 0, \hspace{1ex} y < 0
    \end{cases}
\end{split}
\end{equation}

\noindent Zeigerlänge aus Realteil und Imaginärteil
\begin{equation}
    r = \sqrt{x^2 + y^2}
\end{equation}

\noindent Realteil und Imaginärteil aus Zeigerlänge
\begin{equation}
\begin{split}
    x = r \cdot cos(\varphi) \\
    y = r \cdot sin(\varphi)
\end{split}
\end{equation}

\noindent Winkelgeschwindigkeit, Frequenz und Periodendauer
\begin{equation}
    \omega = 2\pi f = \frac{2\pi}{T}
\end{equation}

\noindent Multiplikation von Potenzen
\begin{equation}
    a^b \cdot a^c = a^{b+c}
\end{equation}



\section{Elektrotechnik}

\subsection{Elementargesetze}

Ohmsches Gesetz (+ im Komplexen)
\begin{equation}
    R = \frac{U}{I} \hspace{3ex}
    \underline{Z} = \frac{\underline{U}}{\underline{I}}
\end{equation}

\noindent Elektrische Leistung
\begin{equation}
    P = U \cdot I = \frac{U^2}{R} = I^2 \cdot R
\end{equation}

\noindent Elektrische Energie
\begin{equation}
    W = P \cdot t
\end{equation}

\noindent Wirkungsgrad
\begin{equation}
    \eta = \frac{P_{in}}{P_{out}}
\end{equation}

\noindent Temperaturabhängigkeit des Widerstands
\begin{equation}
    R_\vartheta = R_{20}\cdot(1 + \alpha_{20} \cdot \Delta \vartheta)
\end{equation}


\subsection{Gleichstromkreise}

Reihenschaltung Widerstände
\begin{equation}
    R_{ges} = \sum_n R_n
\end{equation}

\noindent Parallelschaltung Widerstände
\begin{equation}
\begin{split}
    \frac{1}{R_{ges}} = \sum_n \frac{1}{R_n} \\
    \Rightarrow R_{ges} = \frac{1}{\sum_n \frac{1}{R_n}}
\end{split}
\end{equation}

\noindent Spannungsteiler
\begin{equation}
    \frac{U_1}{U_{ges}} = \frac{R_1}{R_{ges}}
\end{equation}

\noindent Stromteiler
\begin{equation}
    \frac{I_1}{I_2} = \frac{R_2}{R_1}
\end{equation}


\subsection{Wechselstromkreise}

Merksatz Spulen
\begin{quote}
    \textit{,,Bei Induktivitäten, die Ströme sich verspäten.''}
\end{quote}

\noindent Merksatz Kondensatoren
\begin{quote}
    \textit{,,Im Kondensator eilt der Strom vor.''}
\end{quote}

\noindent Widerstand Kondensator
\begin{equation}
    X_C = \frac{1}{j 2\pi f C}
\end{equation}

\noindent Widerstand Spule
\begin{equation}
    X_L = j 2\pi f L
\end{equation}



\section{Energietechnik}


\subsection{Grundlagen}

Minimaler Kurzschlussstrom
\begin{equation}
    I_{kmin} = \frac{c_{min} \cdot U}{\sqrt{3} \cdot R_{min}}
\end{equation}

\noindent Maximaler Kurzschlussstrom
\begin{equation}
    I_{kmax} = \frac{c_{max} \cdot U}{\sqrt{3} \cdot R_{max}}
\end{equation}

\noindent Leiterspannung
\begin{equation}
    U_L = U \cdot \sqrt{3}
\end{equation}


\subsection{Leitungserwärmung}

Stromdichte
\begin{equation}
    S = \frac{I_k}{A_L}
\end{equation}

\noindent Faktor adiabatische Erwärmung (s.u.)
\begin{equation}
    \Psi = \frac{\alpha_{20} \cdot S^2 \cdot t_k}{\gamma_{20} \cdot c}
\end{equation}

\noindent Adiabatische Erwärmung
\begin{equation}
\begin{split}
\vartheta_{Ea} = & \frac{1}{\alpha_{20}} \left[ \left( \left[ 1 + \alpha_{20}(\vartheta_{A} - \vartheta_{20}) \right] e^{\Psi} \right) -1 \right] \\
& + \vartheta_{20}
\end{split}
\end{equation}

\noindent Gesamterwärmung
\begin{equation}
    \vartheta_{E} = \vartheta_{Ea} \cdot \eta_{th}
\end{equation}


\subsection{Kurzschlussstromberechnung}

Kurzschlussimpedanz
\begin{equation}
    Z_k = X_k = X_q = U \cdot \frac{U}{S} = \frac{U^2 }{S}
\end{equation}

\noindent Maximaler Kurzschlussstrom
\begin{equation}
    I_{kmax} = \frac{c_{max} \cdot U}{\sqrt{3} \cdot Z_k}
\end{equation}

\noindent Minimaler Kurzschlussstrom
\begin{equation}
    I_{kmin} = \frac{c_{min} \cdot U}{\sqrt{3} \cdot Z_k}
\end{equation}


\subsection{Impedanzen Quelle und Trafo}

Impedanz Quelle
\begin{equation}
    Z_Q = \frac{c_{max} \cdot U_Q^2}{S_{KQ}} \cdot \frac{1}{\text{Ü}_T^2}
\end{equation}

\noindent Reaktanz Quelle
\begin{equation}
    X_Q = 0,995 \cdot Z_Q
\end{equation}

\noindent Resistanz Quelle
\begin{equation}
    R_Q = 0,1 \cdot X_Q
\end{equation}

\noindent Impedanz Quelle
\begin{equation}
    Z_Q = R_Q + jX_Q
\end{equation}

\noindent Impedanz Trafo
\begin{equation}
    Z_T = \frac{u_{KT}}{100\%} \cdot \frac{U_T^2}{S_T}
\end{equation}

\noindent Resistanz Trafo
\begin{equation}
    R_T = \frac{u_{RT}}{100\%} \cdot \frac{U_T^2}{S_T}
\end{equation}

\noindent Reaktanz Trafo
\begin{equation}
    X_T = \sqrt{Z_T^2 - R_T^2}
\end{equation}

\noindent Impedanz Trafo
\begin{equation}
    Z_T = R_T + jX_T
\end{equation}

\noindent Impedanz Quelle + Trafo
\begin{equation}
    Z_{QT} = (R_Q + R_T) + j(X_Q + X_T)
\end{equation}

\noindent Maximaler Kurzschlussstrom Quelle
\begin{equation}
    I_{KQ} = \frac{c_{max} \cdot U_Q}{\sqrt{3} \cdot Z_{QT}}
\end{equation}

\noindent Gesamter Kurzschlussstrom
\begin{equation}
    I_{K_{gesamt}} = (I_{KQ} + I_{KG} + I_{KM})
\end{equation}


\subsection{Impedanz Netz}

Impedanz Quelle Oberseite
\begin{equation}
    Z_{Q_{OS}} = \frac{c_{max} \cdot U_N}{\sqrt{3} \cdot I_{kQ}} = \frac{c_{max} \cdot U_{NQ}^2}{S_K''}
\end{equation}

\noindent Impedanz Quelle Unterseite
\begin{equation}
    Z_{Q_{US}} = Z_{Q_{OS}} \cdot \frac{1}{\text{ü}^2}
\end{equation}

\noindent Winkel Netz
\begin{equation}
    \varphi_Q = arctan(\frac{X_Q}{R_Q})
\end{equation}

\noindent Reaktanz Quelle Unterseite
\begin{equation}
    X_{Q_{US}} = Z_{Q_{US}} \cdot sin(\varphi_Q)
\end{equation}

\noindent Resistanz Quelle Unterseite
\begin{equation}
    R_{Q_{US}} = Z_{Q_{US}} \cdot cos(\varphi_Q)
\end{equation}

\noindent Spannungsabfall Widerstand
\begin{equation}
    u_r = \frac{P_{cu} \cdot 100\%}{S_{rr}}
\end{equation}

\noindent Spannungsabfall
\begin{equation}
    u_x = \sqrt{u_k^2 - u_r^2}
\end{equation}

\noindent Reaktanz Trafo
\begin{equation}
    X_T = \frac{u_x}{100\%} \cdot \frac{U_{rr}^2}{S_{rr}}
\end{equation}

\noindent Resistanz Trafo
\begin{equation}
    R_T = \frac{u_r}{100\%} \cdot \frac{U_{rr}^2}{S_{rr}}
\end{equation}

\noindent 
\begin{equation}
    x_T = \frac{X_T \cdot S_{rr}}{U_{rr}^2}
\end{equation}

\noindent
\begin{equation}
    K_r = 0,95 \cdot \frac{c_{max}}{1 + 0,6 \cdot x_T}
\end{equation}

\noindent Impedanz Trafo
\begin{equation}
    Z_T = \sqrt{R_T^2 + X_T^2}
\end{equation}

\noindent Gesamtwiderstand
\begin{equation}
    R_{ges} = R_{Q_{US}} + R_{K_{2}} + R_T
\end{equation}

\noindent Gesamtreaktanz
\begin{equation}
    X_{ges} = X_{Q_{US}} + X_{K_{2}} + X_T
\end{equation}

\noindent Gesamtimpedanz
\begin{equation}
    Z_{ges} = \sqrt{R_{ges}^2 + X_{ges}^2}
\end{equation}


\subsection{Impedanz Kabel}

Widerstand Kabel
\begin{equation}
    R_{K_{2}} = R' \cdot l
\end{equation}

\noindent Reaktanz Kabel
\begin{equation}
    X_{K_{2}} = X' \cdot l
\end{equation}

\noindent Maximaler Kurzschlussstrom
\begin{equation}
    I_{kmax} = \frac{c_{max} \cdot U_{RT}}{\sqrt{3} \cdot Z_{ges}}
\end{equation}


\subsection{Kompensation}

Korrigierte Leistung
\begin{equation}
    S = \frac{P}{cos(\varphi)}
\end{equation}

\noindent Zugeführte Leistung
\begin{equation}
    P_{zu} = \frac{P_{ab}}{\eta}
\end{equation}

\noindent Reaktanz
\begin{equation}
    Q = P \cdot cos(\varphi)
\end{equation}

\noindent Winkel
\begin{equation}
    tan(\varphi) = \frac{\Sigma Q}{\Sigma P}
\end{equation}

\noindent Reaktanz
\begin{equation}
    Q_1 = S_1 \cdot sin(\varphi)
\end{equation}

\noindent Reaktanz
\begin{equation}
    Q_C = Q_1 - Q_2
\end{equation}

\noindent Reaktanz
\begin{equation}
    Q = P \cdot (tan(\varphi_1) - tan(\varphi_2))
\end{equation}

\noindent Strom
\begin{equation}
    I_c = \frac{Q}{U_c}
\end{equation}

\noindent Strom
\begin{equation}
    I_c = \frac{U}{X_c} = U \cdot \omega \cdot C
\end{equation}

\noindent Kapazität
\begin{equation}
    C = \frac{I_c}{U \cdot \omega}
\end{equation}

\noindent Kapazität
\begin{equation}
    C = \frac{Q_c}{2 \cdot \pi \cdot f \cdot U^2}
\end{equation}

\noindent Strom
\begin{equation}
    I = \frac{P_{ges}}{U \cdot cos(\varphi)}
\end{equation}



\end{document}
